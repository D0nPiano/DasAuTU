\section{Fazit}

Das Projektseminar hat eine gute Plattform vorgegeben, die viele M"oglichkeiten zur Bew"altigung der Aufgabenstellung 
bietet. Insbesondere die Kinect-Kamera stellt eine sehr
starke Erweiterung der Hardware dar und macht das Navigieren mit vorhandenen ROS Paketen relativ einfach.
Wir konnten sie aber auch gut in unsere eigenen Paketen einbinden. Die Simulation ist ein hilfreiches Werkzeug, um zu testen, ob Code grunds"atzlich funktioniert, allerdings gab es eine starke Abweichung zum realen Verhalten des Fahrzeugs.		

 

\subsection{Probleme}

Wie einige andere Gruppen hatten auch wir das Problem, dass der Mikrocontroller h"aufig abgest"urzt ist. Bei einem Absturz konnten Motor- und Lenkbefehle nicht mehr versendet werden, was zu Unf"allen f"uhren konnte und das Testen erschwert hat.
Die Ursache blieb bis zum Schluss unklar. Allerdings konnten wir das Problem durch das Verwenden einer "alteren Firmware des Mikrocontroller-Boards lösen.



\subsection{Ausblick und m"ogliche Verbesserungen}

Unsere Gruppe musste viel Zeit investieren, um sich in die Grundlagen von der Verwendung von ROS und des Basis Pakets einzuarbeiten. 
Durch eine ausf"uhrlichere Einf"uhrung in ROS und Dokumentation des Basis Pakets k"onnten die Gruppen sich mehr auf die eigentliche Aufgabenstellung
konzentrieren.

Um das Projektseminar noch weiter voranzubringen, halten wir es f"ur sinnvoll, mehr auf den L"osungen der Vorjahresgruppen aufzubauen.
Da sich durch die Verwendung der Kinect eine Navigation mithilfe der Pakete AMCL und teb local planner anbietet, welche auch von vielen Gruppen verwendet wurden,
k"onnte man dies in das Basis Paket aufnehmen. So k"onnte man in Zukunft neue und komplexere Aufgaben bew"altigen.
Au{\ss}erdem k"onnte man Gruppen mehr dazu auffordern, sich eigene Aufgabenstellungen auszudenken, um noch mehr einzigartige Ideen im Projektseminar zu gestalten.
