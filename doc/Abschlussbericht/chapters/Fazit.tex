\section{Fazit}

Das Projektseminar hat eine gute Plattform vorgegeben die viele M"oglichkeiten zur Bew"altigung der Aufgabenstellung 
bietet. Die Kinect Kamera ist eine sehr
starke Erweiterung der Hardware und macht das Navigieren mit vorhandenen ROS Paketen relativ einfach.
Wir konnten sie aber auch gut in unseren eigenen Paketen einbinden. Die Simulation war ein hilfreiches Werkzeug um zu testen ob Code grunds"atzlich funktioniert, allerdings gab es eine starke Abweichung zum realen Verhalten des Roboters.		

 

\subsection{Probleme}

Wie einige andere Gruppen hatten wir das Problem das der Mikrocontroller h"aufig abgest"urzt ist. Bei einen Absturz konnten Motor und Lenkung nicht mehr verwendet werden, was "ofters zu Unf"allen gef"uhrt hat und das Testen erschwert hat.
Die Ursache blieb bis zum Schluss unklar. Allerdings konnten wir das Problem durch das Verwenden einer "alteren Firmware l"osen.



\subsection{Ausblick und m"ogliche Verbesserungen}

Unsere Gruppe musste viel Zeit investieren um sich in die Grundlagen von der Verwendung von ROS und des Basis Pakets einzuarbeiten. 
Durch eine ausf"uhrlichere Einf"uhrung in ROS und Dokumentation des Basis Pakets k"onnten die Gruppen sich mehr auf die eigentliche Aufgabenstellung
konzentrieren.

Um das Projektseminar noch weiter voranzubringen halten wir es f"ur sinnvoll mehr auf den L"osungen der Vorjahresgruppen aufzubauen.
Da sich durch die Verwendung der Kinect eine Navigation mithilfe der Pakete AMCL und teb local planner anbietet, und auch von vielen Gruppen verwendet wurde,
k"onnte man dies in das Basis Paket aufnehmen. So k"onnte man in Zukunft neue und komplexere Aufgaben bew"altigen.
Au{\ss}erdem kann man Gruppen mehr dazu auffordern sich eigene Aufgabenstellungen auszudenken um noch mehr einzigartige Ideen im Projektseminar zu finden.
