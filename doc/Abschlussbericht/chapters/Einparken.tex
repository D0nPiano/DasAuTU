\section{Paralleles Einparken}
	Bei dieser Aufgabe muss das Auto in eine m�glichst kleine Parkl�cke hinter einem anderen bereits geparkten Fahrzeug r�ckw�rts parallel einparken. Die theoretischen Grundlagen der in diesem Abschnitt gemachten Annahmen entstammen der Dissertation HIER QUELLE EINF�GEN. Die Herleitung der Formeln kann der Quelle entnommen werden.

  \subsection{Grundidee}
  Das Fahrzeug f�hrt zu einer zuvor berechneten Startposition. Dort angekommen wird nach rechts eingelenkt und r�ckw�rts gefahren. Wird nun ein gewisser Gierwinkel erreicht, wird nach links gelenkt bis das Fahrzeug in der Parkl�cke steht. 
  
  \subsection{Ackermann-Lenkung}
  Als erstes ist es wichtig zu verstehen, wie sich das Auto bei Lenkbewegungen verh�lt. Dazu greift man auf das Modell der Ackermann-Lenkung zur�ck.
  Das Auto bewegt sich beim Kurven fahren auf einer Kreisbahn, wobei s�mtliche R�nder im 90� Winkel zum Rotationszentrum stehen. Das Rotationszentrum liegt also im hinteren Bereich des Autos auf H�he der hinteren R�der.
  
  
  \subsection{minimale Parkl�cke}
  In der Aufgabenstellung ist gefordert, dass die Parkl�cke m�glichst klein sein soll. Was ist also die minimal ben�tigte Parkl�ckengr��e bei gegebenem Kurvenradius des Fahrzeugs? Besagte Gr��e l�sst sich mit einem Gedankenexperiment recht leicht ermitteln. Hierf�r wird angenommen, dass sich das Auto bereits in der optimalen Parkl�cke befindet. In diesem Fall gelingt es dem Auto gerade noch so bei voll eingeschlagener Lenkung aus der L�cke heraus zu fahren, ohne andere Objekte zu ber�hren.
  
  \subsection{Parkl�ckenerkennung}
  Unsere Annahme ist, dass sich vor uns im Sichtbereich der Kinect-Kamera ein geparktes Auto befindet, an dem man sich orientieren kann. Wir erkennen die hintere linke Ecke des geparkten Autos, indem wir die Punkte des Laserscans mittels Linienerkennung nach 2 Linien absuchen, die zu einer Ecke passen.
  
  \subsection{Berechnung des maximalen Gierwinkels}
  \[R_e=\sqrt{R^2+b^2}\]
  \[\alpha = \arccos(1-\frac{b^2}{4R^2})\]
  \[\beta = \arcsin(\frac{R}{R_e}*\sin(\alpha))\]
  \[\theta = \arccos(\frac{R-W}{R_e})-\beta\]	
  Verwendete Variablen:\begin{itemize}
  \item R: minimaler Kurvenradius
  \item W: Breite des Autos
  \item b: Distanz zwischen Hinterrad und vordere Ecke des Autos
  \item $ \theta: $ Gierwinkel
  \end{itemize}
  
  \subsection{Sicherheitsabstand} 
  Die bisherigen Annahmen wurden unter idealen Bedingungen gemacht, sodass das Auto etwa mit Abstand 0 an anderen Objekten vorbei f�hrt. Da dies in echt nicht praktikabel ist, berechnen wir alle Werte mit einem zus�tzlichen Sicherheitsabstand. Dies wird erreicht, indem wir sowohl unser Auto breiter annehmen als es eigentlich ist und noch die Startposition f�r die Einlenkbewegung etwas verschieben.  