\section{Rundkurs mit Hindernissen}
	Bei dieser Aufgabe muss das Auto einen bekannten Rundkurs im Flur des Instituts in 	m�glichst kurzer Zeit bew�ltigen. Allerdings befinden sich auf der Strecke nicht verzeichnete Hindernisse, die dynamisch erkannt und umfahren werden m�ssen.

  \subsection{Grundidee}
    
  \subsection{Alternative (verworfene) L�sung}
  	Im Laufe des Projekts haben wir auch eine andere L�sung f�r den Rundkurs mit Hindernissen programmiert, die jedoch sp�ter wieder verworfen wurde. Der Algorithmus besteht generell aus zwei Teilen:
  	Beim ersten Teil sucht sich das Auto mithilfe der Kinect Kamera den Punkt, der am weitesten von dem Auto entfernt ist. Dann steuert das Auto gerade auf diesen Punkt zu. Um jedoch um Kurven fahren zu k�nnen, braucht das Auto einen Linksdrall, da es ansonsten einfach an der Kurve geradeaus den Gang herunter fahren w�rde. Dazu wird eine Funktion auf alle Laserscan-Punkte angewendet, die in Abh�ngigkeit zum Winkel zur Fahrtrichtung den gemessenen Abstand verringert, je weiter rechts desto mehr. Au�erdem wurde die Funktion bei gro�en Ultraschall-Abstandswerten nach links verst�rkt, um nochmals besser um Kurven herum zu kommen.
  	Dieser Ansatz hat beim Hindernis umfahren und Kurven fahren gut funktioniert, jedoch f�hrte der Linksdrall auf geraden zu stark schwankender Fahrt, da das Auto immer wieder auf die Wand zusteuerte und dann wieder ausweichen musste. Aus diesem Grund haben wir den Algorithmus um den zweiten Teil, einem PD-Regler, erweitert. Dieser funktionierte im Prinzip bis auf leicht andere Parameter genauso wie bei dem Rundkurs ohne Hindernisse: Der Regler h�lt den Abstand zur linken Wand auf einem konstanten Niveau, solange kein Hindernis von der Kinect Kamera vor dem Auto in einem bestimmten Abstand erkannt wurde. Diese Hinderniserkennung suchte ein etwa 1,2 Meter langes und 0,4 Meter breites Rechteck vor dem Auto nach Laserscan Punkten ab, die ein Hindernis andeuten. Sollte solch ein Hindernis erkannt werden, wurde der Regler deaktiviert und der weiteste-Distanz-Controller benutzt.
  	Der Regler war allerdings nicht stark genug, um um enge Kurven herum zu fahren. Darum haben wir wiederum die Kurvenerkennung aus dem Rundkurs ohne Hindernisse benutzt, um den Regler f�r einige Sekunden zu deaktivieren und die Kurve mit dem weiteste-Distanz-Algorithmus zu fahren. Die Kurvenerkennung war jedoch aufgrund der unregelm��igen Positionierung von Hindernissen und dem unregelm��igen Abstand zur Wand nicht mehr ganz zuverl�ssig, insbesondere false-positive-Fehler traten immer wieder auf. Dies war jedoch noch verschmerzbar, da der weiteste-Distanz-Controller generell auch gut auf der Geraden fahren kann.
  	Letztendlich wurde der Algorithmus aber aufgrund verschiedener kleiner Probleme verworfen, insbesondere das fehlende Ged�chtnis war ein Problem (also dass das Auto sich bereits vorher erkannte Hindernisse nicht merkt und es auch keinerlei Wissen �ber die aktuelle Position besitzt). 