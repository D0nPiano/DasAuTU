\section{Rundkurs ohne Hindernisse}
	Bei dieser Aufgabe muss das Auto einen bekannten Rundkurs im Flur des Instituts in 	m�glichst kurzer Zeit bew�ltigen. 

  \subsection{Grundidee}
    Der Rundkurs besteht aus Geraden, an deren Ende sich jeweils eine Kurve befindet. Demnach existieren in diesem System zwei Hauptzust�nde: \glqq Fahrzeug auf der Geraden\grqq ~oder \grqq Fahrzeug in Kurve\grqq. Die Modellierung des Ganzen als Zustandsautomat erscheint also sinnvoll.

  \subsection{Regelung der Fahrt auf den Geraden}
	Jede Gerade hat den Vorteil, dass sie durch eine Wand begrenzt ist, an welcher sich das Fahrzeug orientieren kann. Der aktuelle Abstand zur Wand kann n�mlich n�herungsweise mit dem Ultraschall-Sensor, der auf der linken Seite des Autos montiert ist, bestimmt 		werden. Mit Hilfe eines PD-Reglers wird sichergestellt, dass das Auto sich parallel zur 	Wand fortbewegt.
	\\
	Die dazu ben�tigten Parameter wurden bei Testfahrten empirisch ermittelt. Au�erdem 		werden die Messwerte des Ultraschall-Sensors gegl�ttet, um nerv�se Lenkbewegungen zu 		vermeiden, indem aus den letzten 10 Messwerten ein Mittelwert gebildet wird.      

  \subsection{Steuerung der Fahrt in den Kurven}
  	Zu Beginn wurde auch hier versucht, das Problem mittels einer Regelung zu l�sen. Mit der Zeit sollte sich jedoch herausstellen, dass eine simple Steuerung bessere Ergebnisse 	erzielt. In der Kurve wird also einfach ein fester Lenkwinkel eingestellt, der w�hrend 		der gesamten Kurvenfahrt konstant bleibt.
  	
  \subsection{Kurvenerkennung}
  	Die Kurvenerkennung entspricht im Zustandsautomaten dem �bergang von der Geraden hin zur	Kurve. Wie erkennt man also eine Kurve? Verl�sst man sich ausschlie�lich auf den 		Ultraschall-Sensor, kann man eine Kurve erst detektieren, wenn sich links des Fahrzeugs 	keine Wand mehr befindet. Dann befindet sich das Auto allerdings schon mitten im 			Kurvenbereich und eine vern�nftige Linie kann keinesfalls mehr erreicht werden.
  	\\
	An dieser Stelle kommt die Kinect-Kamera ins Spiel. Durch sie ist es m�glich, bereits 	deutlich vor einer anstehenden Kurve selbige zu erkennen. Somit wird es �berhaupt erst 		m�glich eine optimale Kurvenlinie zu fahren.
	\\
	Auf diesem Rundkurs zeichnet sich eine Kurve dadurch aus, dass am Kurvenbeginn die Wand abrupt endet. Die gemessenen Distanzen im von der Kinect generierten Laserscan steigen an dieser Stelle also sprunghaft an.  
  
  \subsection{Kurvenende}	
  	Der vorgegebene Rundkurs enth�lt ausschlie�lich 90�-Kurven. Am Kurvenausgang ist demnach der Gierwinkel des Fahrzeugs etwa 90� gr��er als am Kurveneingang. Steigt der Gierwinkel also �ber einen gewissen Schwellwert, �bernimmt der PD-Regler wieder die Kontrolle.
  	
  \subsection{Probleme}
  \subsubsection{Kurvenerkennung am Kurvenausgang}
  WTF
  
    