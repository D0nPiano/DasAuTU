\section{Projektstruktur}
\label{sec:projektstruktur}

\subsection{Hauptprogramm}
Um einen einfachen �bergang von einem Modus in einen anderen zu erm�glichen, entschieden wir uns daf�r, nicht jeden Modus als eigenst�ndige Node zu schreiben, sondern eine Hauptnode zu erstellen, die im jeweiligen Modus dann nur den dazugeh�rigen Controller erstellt.

\paragraph{}
Die Controller implementieren hierbei jeweils eine Run-Methode und kann direkt auf den Command-Publisher der Hauptnode zugreifen. Kommandos werden nur von der Run-Methode gesendet. Dies bietet einerseits den Vorteil, dass Kommandos nur mit einer in der Hauptnode definierten, bestimmten Frequenz an das Microcontroller Board gesendet werden und andererseits die M�glichkeit, Controller f�r Zeitr�ume die Publish-Befugnis zu entziehen. Dadurch ist es zum Beispiel m�glich, eine globale Notbremse zu implementieren oder global einen Controller anzuschlie�en, der immer die Daten des jeweiligen Modus �berschreibt.

\paragraph{Fazit}
Das Prinzip des �bergeordneten Hauptprogramms erm�glichte uns in der Praxis, sehr simpel globale �nderungen an z.B. der Notbremse vorzunehmen. Auch eine Controller�bergreifende Nutzung von Daten wurde dadurch m�glich, in dem beispielsweise die Hauptnode Laserscans abboniert, und der Pointer an die Controller �bergeben wird. Neue Controller m�ssen so nicht erst warten bis n�chster Laserscan gesendet wird, sondern k�nnen bereits beim Moduswechsel auf die in der Hauptmethode gespeicherten Daten zugreifen. 